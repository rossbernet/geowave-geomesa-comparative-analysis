\subsection{Generality of the Architecture}
\label{sec:featurecompare:generality}

A major difference between the projects is the generality of the architectures when it comes to supporting
various backends\footnote{Backend refers to the specific mechanism for storing vector, data and the related geospatial index (ex: Accumulo, HBase).}
and indexing strategies.
GeoWave API has a focus on being an N-Dimensional indexing mechanism for arbitrary backends.
The fact that this document focuses on its ability to handle geospatial data ($2$-Dimensional spatial and $3$-Dimensional spatiotemporal data) is only based on the currently known GeoWave use cases.
However, the project aims at supporting data with arbitrary dimensionality.
GeoMesa API, on other hand, is directly tailored to geospatial indexing allowing for clearer interface, implementation, and paths for optimization.

GeoWave specifically is designed around abstractions that remain agnostic about the storage and access implementations.
This could provide more flexibility for developing backend support, which might explain why GeoWave HBase support is more mature than GeoMesa's.
GeoMesa focuses on using GeoTools' abstractions, and thus is more dependent on GeoTools as a base library.
GeoMesa also focuses less on dealing with abstractions; this may have an effect that features written for one backend are difficult to translate to another backend.
However, dealing with less abstraction can be more straightforward, and some developers may find it easier to understand and work with the GeoMesa API.
